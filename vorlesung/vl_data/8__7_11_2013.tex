\datum{7.11.2013}


Horn-Formeln\\
VKNF mit $\leq 1$ ps-Literal pro Disjunktionsterm\\

$(A_i \OR A_{ij} \OR … \OR A_J) \rightsquigarrow (A_{j1} \AND … \AND A_{js} \rightarrow  A_i)$\\
$A_i \rightsquigarrow (W \rightarrow A_i)$\\
$\neg A_{j1} \OR … \OR A_{js} \rightsquigarrow (A_{j1} \OR … \OR A_{js} \rightarrow F)$\\

\satz{} Markierungsalgorithmus korrekt\\

\beweis{}
$G$ Hornformel und $G$ erfüllbar\\
$\Rightarrow f(A_i) = W$ für alle $A_i$\\
din im Markierungsalgorithmus markiert werden.\\

$\rightarrow$ ist $G$ erfüllbar, so wird dies vom Markierungsalgorithmus festgestellt\\

\annahme{}
Markierungsalgorithmus kommt zum Ergebnis unerfüllbar\\
Dann gibt es nach einigen Schritten im Markierungsalgorithmus eine Formel $(A_{j1} \AND … \AND A_{ji} \rightarrow F)$ wobei $A_{j1}, …, A_{js}$ markiert sind\\

\behauptung{}
Für jede Interpretation $f$ mit $f(G) = W$ gilt\\
$f(A_{j1}) = … = f(A_{js}) = W$\\
$G$ enthält aber $((A_{j1} \AND … \AND A_{js}) \rightarrow F) \rightsquigarrow ( \neg A_{j1} \OR … \OR \neg A_{js})$\\
Da $f(\neg A_{j1} \OR … \OR \neg A_{js}) =F$ folgt $f(G)=F$\\
$\Rightarrow$ also Annahme falsch und Markierungsalgorithmus kommt zum Ergebnis „erfüllbar“\\

$\rightarrow$ Ist $G$ unerfüllbar, so wird das im Markierungsalgorithmus festgestellt.\\

\annahme{}
Markierungsalgorithmus kommt zum Ergebnis erfüllbar.\\
Definieren Interpretation $f$ durch\\
$f(A_i) = \begin{cases} W\ A_i\ bei\ Abbruch\ markiert \\ F\ sonst \end{cases}$\\

Betrachten die verschiedenen Disjunktionsterme von $G$\\
$(A_i \rightarrow W)$ Dann ist $A_i$ im 1 Schritt markiert und $f(A_i) = W$\\
$(A_{j1} \AND … \AND A_{js} \rightarrow A_i)$\\

\begin{minipage}{\textwidth}
Entweder\\*
$A_{j1}, …, A_{js}$ und $A_i$ markiert\\*
Dann $f((A_{j1} \AND … \AND A_{js}) \rightarrow A_i) = W$ da\\*
$f(A_{j1}) = … = f(A_{js})=W$\\*

oder es gilt $1 \leq l \leq s$, so dass $A_{jl}$ nicht markiert.\\*
Also $f(A_{jl}) = F$\\*
$\Rightarrow f((A_{j1} \AND … \AND A_{jl}) \rightarrow A_i) = W$\\*
\end{minipage}

$(A_{j1} \AND … \AND A_{js} \rightarrow F)$\\
Da der Algorithmus mit „erfüllbar“ abbricht gibt es $1 \leq l \leq s$ mit $A_{jl}$ nicht markiert. Also $f(A_{jl})=F$\\
Damit $f(\neg A_{j1} \OR … \OR \neg A_{js}) = W$\\
$\Rightarrow f$ erfüllt alle Disjunktionsterme von $G$\\
$\Rightarrow f(G) = W  \rightarrow$ \widerspruch zur Annahme\\
$\Rightarrow$ Markierungsalgorithmus kommt zum Ergebnis „unerfüllbar“ $\Box$\\

\definition{}
Seien $f,g$ Modell von einer Formel $G$
\begin{itemize}
\item Wir sagen $f \leq g \Leftrightarrow (f(A_i) = W \Rightarrow g(A_i) =W$ für alle atomaren Formeln $A_i$ aus $G$)

\item Wir sagen $f$ ist minimales Modell von $G$ falls $f \leq g$ für jedes Modell $g$ von $G$
\end{itemize}

\beispiel{}
$G = A_1 \AND \neg (\neg A_2 \OR A_3)$\\
Modelle\\
\begin{tabular}{c|c|cr}
$A_1$ & $A_2$ & $A_3$ &  \\
\hline
W & F & F & $f_1$ \\
W & F & W & $f_2$ \\
W & W & W & $f_3$ \\
\end{tabular}
(minimales Modell (alle anderen Modelle sind größer oder gleich))\\
$f_1 \leq f_2$\\
$f_1 \leq f_3$\\
$f2 \leq f_3$\\

\folgerung{2.8}
Sei $G$ eine erfüllbare Horn-Formel, dann besitzt $G$ ein minimales Modell

\beweis{}
Im Beweis von Satz 2.7 wird gezeigt:
\begin{itemize}
\item[$\rightarrow$] Alle im Markierungsalgorithmus markierten atomaren Formeln müssen durch ein Modell von $G$ als W interpretiert werden

\item[$\rightarrow$] Die Interpretation $f$, die alle im Markierungsalgorithmus markierten atomaren Formeln mit W und alle anderen als F interpretiert ist ein Modell von $G$

\item[$\rightarrow$] Dieses $f$ ist das minimale Modell $\Box$
\end{itemize}

\bemerkung{}
Aus Korrektheit des Markierungsalgorithmus folgt
\begin{enumerate}
\item[(i)] Eine Horn-Formel mit genau einem pos. Literal pro Disjunktionsterm ist erfüllbar

\item[(ii)] Eine Horn-Formel ohne einen Disjunktionsterm, der nur aus einem positiven Literal besteht ist erfüllbar.
\end{enumerate}

\folgerung{2.9}
Für Horn-Formeln kann die Erfüllbarkeit in polynomialer Zeit entschieden werden.

\beweis{}
Analyse des Markierungsalgorithmus\\
Schritt (1) lineare Zeit\\
Schritt (2) lineare Zeit (allerdings wird dieser Schritt öfters gemacht)\\
Schritt (2) wird $\leq$ Formellänge oft durchlaufen (worst-case)\\
$\Rightarrow$ quadratische Laufzeit $\Box$

\newpage

\section{Kompaktheitssatz}

Wiederholung\\
$\Sigma$ Formelmenge heißt erfüllbar: $\Leftrightarrow$ Es gibt Interpretation $f$ mit $f(G) = W$ für $G \in \Sigma$\\
(so ein $f$ heißt Modell von $\Sigma$)\\

Falls $\# \Sigma < \infty$ und $\Sigma = \{ G_1, …, G_r\}$\\
$\Sigma$ erfüllbar $\Leftrightarrow G_1 \AND … \AND G_r$ erfüllbar\\

\satz{3.1} (Kompaktheitssatz)\\
Sei $\Sigma$ Formelmenge. Dann gilt:\\
$\Sigma$ erfüllbar $\Leftrightarrow$ jede endliche Teilmenge von $\Sigma$ ist erfüllbar.

\beweis{}
\begin{itemize}
\item[„$\Rightarrow$“] $\Sigma$ erfüllbar $\Rightarrow$ Es gilt Interpretation $f$ mit $f(G) =W$ für alle $G \in \Sigma$\\

Sei $\Sigma' \subseteq \Sigma$ endlich\\
Dann $f(G) = W$ für alle $G \in \Sigma'$\\
$\Rightarrow \Sigma'$ erfülbar

\item[„$\Leftarrow$“] Sei $\Sigma$ Formelmenge über $\{A_1, A_2, … \}$\\
$\Sigma_n := \{ G \in \Sigma$ | keine atomare Formel $A_i$ mit $i \gneq n$ kommt in $G$ vor $\}$

\end{itemize}

Es gibt $2^{2^n}$ semantische Klassen von Formeln über $\{A_1, …, A_n\}$\\
$\Rightarrow \Sigma'_n =$ Menge von Repräsentationen der semantischen Klassen, für die es eine Formel in $\Sigma_n$ gibt, ist endliche Menge, $\# \Sigma'_n < \infty$\\
$\Rightarrow \Sigma'_n$ erfüllbar\\

Jedes Modell von $\Sigma'_n$ ist Modell von $\Sigma_n$\\
$\Rightarrow$ Es gibt Interpretation $f_n, n \geq 1$, mit $f_n(G) = W$ für alle $G \in \Sigma_n$\\
Es gibt $\Sigma_1 \subseteq \Sigma_2 \subseteq … \subseteq \Sigma$\\
Konstruktion einer Interpretation $f$ mit $f(G)=W$ für alle $G \in \Sigma$
\begin{enumerate}
\item[(1)] Indexmenge $I = \{ 1, 2, 3, …\}, i:=1$
\item[(2)] Falls $\# \{ n \in I\ |\ f_n(A_i) = W \} = \infty$\\
dann $f(A_i) = W$\\
$I := I \backslash \{ n\ |\ f_n(A_i)=F \}$\\
sonst\\
$f(A_i) = F$\\
$I := I \backslash \{ n\ |\ f_n(A_i)  W\}$
\item[(3)] $i=i+1$, gehe zu (2)
\end{enumerate}