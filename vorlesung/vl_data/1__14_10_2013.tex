% VL vom 14.10.2013

\section{Aussagenlogik}

\begin{flushright}
\textit{Vorlesung vom 14.10.2013}
\end{flushright}

Logische Systeme haben immer zwei Teile:
\begin{itemize}
\item Syntax (Formeln)
\item Semantik (Wahrheitswerte der Formeln)
\end{itemize}

\definition{(Syntax der Aussagenlogik)}
\begin{itemize}
\item Menge $A = \{ A_1, A_2, ... \}$ von atomaren Formeln
\item Wir definieren induktiv die Menge der Formeln der Aussagenlogik
    \begin{itemize}
        \item jede atomare Formel ist Formel
        \item sind $G$ und $H$ Formel, so auch \\
$( G \AND H )$ „$G$ und $H$“ \\
$(G \OR H)$ „$G$ oder $H$“ \\
$\neg G$ „nicht $G$“
    \end{itemize}
\end{itemize}

\bemerkung{}
Die Menge der Formeln der Aussagenlogik ist Sprache (= Menge von Wörtern) über Alphabet $A \cup \{\AND, \OR, \neg, (, ) \}$\\

\beispiel{}
\begin{itemize}
\item $ \left( ( A_{17} \OR A_2 ) \AND \neg (A_3 \AND A_4) \right)$\\
    Formel der Aussagenlogik
\item reine syntaktische Objekte\\
    $\neg \neg A \neq A$
\end{itemize}

\beweis{}
Sei $G$ eine Formel der Aussagenlogik\\
$T(G)$ die Menge der Teilformeln von $G$ induktiv definiert als\\
$T(G) = \{ G \}$ falls $G$ atomare Formel\\
$T(G) = T(G_1) \cup T(G_2) \cup \{G\}$, falls $G = (G_1 \OR G_2)$ oder $G=(G_1 \AND G_2)$\\
$T(G)=T(G_1) \cup \{G\}$\\

\beispiel{}
$G= \left( (A_{17} \OR A_2) \OR \neg (A_3 \AND A_4) \right)$\\
$T(G) = T \left( (A_{17} \OR A_2) \right) \cup T \left(\neg ( A_3 \AND A_4) \right) \cup \{G\}$\\
$= \{ A_{17}, A_2, (A_{17} \OR A_2) \} \cup T((A_3 \AND A_4)) \cup \{ \neg (A_3 \AND A_4) \} \cup \{G\}$\\
$= \{A_{17}, A_2, (A_{17} \OR A_2), A_3, A_4, (A_3 \AND A_4), \neg (A_3 \AND A_4), \left( (A_{17} \OR A_2) \AND \neg (A_3 \AND A_4) \right) \}$\\
    
\bemerkung{}
$T(G)$ ist die Menge der Formeln, die bei der induktiven Konstruktion von $G$ auftauchen.\\
\noindent\\
Sprechweisen:
%\begin{tabbing}
\begin{itemize}
\item Für $(G \AND H)$ sagt man auch „Konjunktion von $G$ und $H$“
\item Für $(G \OR H)$ sagt man auch „Disjunktion von $G$ und $H$“
\item Für $\neg G$ sagt man auch „Negation von G“.
\end{itemize}
%\end{tabbing}

\vspace{1cm}
\noindent
Abkürzende Schreibweisen:\\
\noindent\\
$G, H$ Formeln der Aussagenlogik
\begin{itemize}
\item $(G \rightarrow H)$ aus $G$ folgt $H$, für $(\neg G \AND H)$; $G$ impliziert $H$; Implikation
\item $(G \leftrightarrow H)$ $G$ äquivalent zu $H$, für $(G \rightarrow H) \AND (H \rightarrow G)$; Äquivalenz
\item $( \bigvee_{i=1}^{n} G_i )$ für $\left( … (G_1 \OR G_2) \OR G_3 ) … \OR G_n \right)$ \hspace{1cm}mit $G_1, …, G_n$ Formeln
\item $( \bigvee_{i=1}^n G_i )$ für $\left(… ( G_1 \AND G_2) \AND G_3) … \AND G_n \right)$ \hspace{1cm}mit $G_1, …, G_n$ Formeln
\end{itemize}

\vspace{1cm}
\definition{Semantik der Aussagenlogik}
\begin{itemize}
\item Sei $\emptyset \neq A' \subseteq A$ eine Teilmenge der atomaren Formeln
\item Eine Abbildung $f: A' \rightarrow \{W,F\}$ \hspace{1cm} (wahr, falsch)\\
    heißt Interpretation von $A'$
\item Eine Formel $G$ heißt Formel über $A'$ falls $T(G) \cap A \subseteq A'$
\item eine Interpretation $f: A \rightarrow \{W, F\}$ heißt passend zu Formel $G$, falls $G$ Formel über $A'$ ist.
\item Sei $f: A' \rightarrow \{W, F\}$ eine zur Formel $G$ passende Interpretation. Dann definieren wir $f(G)$ induktiv.\\
    $f(G) = f(( G_1 \AND G_2 )) = \begin{cases} W\ falls\ f(G_1) = f(G_2) = W \\ F\ sonst\end{cases}$\\
    $f(G) = f((G_1 \OR G_2)) = \begin{cases} W\ sonst \\ F\ falls\ f(G_1)=f(G_2)=F \end{cases}$\\
    $f(G) = f(\neg G_1) = \begin{cases} W\ falls\ f(G_1)=F \\ F\ falls\ f(G_1)=W \end{cases}$\\
    für $G=\neg G_1$
\end{itemize}

\beispiel{Semantik der abkürzenden Schreibweisen}

\begin{tabular}{c|c|c|c}
$G$ & $H$ & $(G \rightarrow H)$ & $(G \leftrightarrow H)$\\
\hline
F & F & W & W\\
F & W & W & F\\
W & F & F & F\\
W & W & W & W\\
\end{tabular}

\lemma{1.1}
Sei $A'$ eine Menge von $n$ atomaren Formeln. Dann gibt es $2^n$ Interpretationen von $A'$\\

\beweis{}
Für jede atomare Formel mit $A'$ gibt es $2$ Möglichkeiten für das Bild unter $f$\\
$\rightarrow$ Die atomaren Formeln $A'$ können unabhängig voneinander interpretiert werden\\
$\Rightarrow$ $\#$ Interpretationen $=\underbrace{2 \cdot 2 \cdot 2}_n = 2^n$ \hspace{1cm}\ensuremath{\Box}\\

\definition{}
Sei $G$ eine Formel über $A' = \{A_1, …, A_n \}$\\
seien $f_1, …, f_{2^n}$ die $2^n$ Interpretationen von $A'$\\
Dann heißt das Schema\\
\begin{tabular}{c|c|c|c}
$A_1$ & … & $A_n$ & $G$\\
\hline
$f_1 (A_1)$ & … & $f_1 (A_n)$ & $f_1 (G)$\\
\vdots & & \vdots & \\
$f_{2^n} (A_1)$ & … & $f_{2^n} (A_n)$ & $f_{2^n} (G)$
\end{tabular}\\
Wahrheitstabelle von $G$\\

\beispiel{}
$(A_1 \OR (A_2 \AND \neg A_3))$\\

\begin{tabular}{c|c|c|c}
$A_1$ & $A_2$ & $A_3$ & $(A_1 \OR (A_2 \AND \neg A_3))$\\
\hline
F & F & F & F\\
F & F & W & F\\
F & W & F & W\\
W & F & F & W\\
F & W & W & F\\
W & F & W & W\\
W & W & F & W\\
W & W & W & W\\
\end{tabular}\\
    
\definition{}
Sei $F_{A'}$ die Menge aller Formeln über $A'$\\
Für $G, H \in F_{A'}$ sagen wir $G$ ist semantisch äquivalent zu $H$, falls $f(G) = f(H)$ für alle Interpretationen $f:A' \rightarrow \{W,F\}$\\
Wir schreiben $G \equiv H$\\