\datum{11.11.2013}


Kompaktheitssatz\\
$\Sigma$ erfüllbar $\Leftrightarrow$ Jede endl. Teilmenge von $\Sigma$ erfüllbar.\\

\beweis{}
„$\Rightarrow$“\\
„$\Leftarrow$“ $\Sigma_n$ = Formeln aus $\Sigma$ über $A_1, …, A_n$\\
$\Sigma'_n =$ Repräsenten der semantischen Klassen aus $\Sigma_n$\\

Fakt: $\# \Sigma'_n < \infty$\\
Daher $\Sigma'_n$ erfüllbar und es gibt Modell $f_n$

\begin{enumerate}
\item[(1)] $I = \{1, 2, … \}, i=1$

\item[(2)] $\# \{ n \in I | f_n(A_i)=W \} = \infty \rightsquigarrow f(A_i) = W$\\
$I= I \backslash \{ n | f_n(A_i) = F \}$\\
sonst $f(A_i)=F, I:= I \backslash \{ n | f_n(A_i) = W \}$

\item[(3)] $i=i+1$ gehe zu (2)
\end{enumerate}

Wir zeigen: $f$ ist Modell für $G$\\

\behauptung{}
Sei $I$ die Menge der Indizes nach dem $i$-ten Durchlauf von (2)-(3).\\
Dann gilt $\# I = \infty$ und für alle $m \in I$ ist $f_m (A_j) = f(A_j), 1 \leq j \leq i$\\

\beweis{}
Induktion nach $i$\\
I.A.:\\
$i=0$\\
$\# I = \# \{ 1, 2, … \} = \infty$\\
und $f(A_j) = f_m(A_j)$ für $1 \leq j \leq i$ ist leere Bed.\\

I.S.:\\
$i \rightarrow i+1$\\
Sei $I$ die Menge nach dem $(i+1)$-ten Durchlauf und $I'$ die Menge nach dem $i$-ten Durchlauf\\

\underline{1. Fall} $\# \{ n \in I' | f_n(A_{i+1}) = W \} = \infty$\\
$\rightarrow I = I' \backslash \{ n \in I' | f_n(A_{i+1}) = F \}$\\
$= \{ n \in I' | f_n(A_{i+1}) = W \}$\\
$\Rightarrow \# I=\infty$\\

Nach I.V. gilt $f_m(A_j) = f(A_j), 1 \leq j \leq i$\\
und $m \in I'$\\
Da $I \subseteq I'$ folgt\\
$f_m(A_j) = f(A_j), 1 \leq j \leq i$ und $m \in I$\\
Im 1. Fall wird $f(A_{i+1})=W$ und $f_m(A_{i+1})=W$ für $m \in I$\\
$\Rightarrow f(A_j) = f_m(A_j)$ für $1 \leq j \leq i+1$ und $m \in I$\\
\underline{2. Fall} $\# \{ n \in I' | f_n(A_{i+1})=W \} < \infty$\\
$\Rightarrow$ Da nach I.V. $\# I' = \infty \Rightarrow \# \{n \in I' | f_n(A_{i+1}) = F \} = \infty$\\

Beweis nun analog zum 1. Fall\\
Sei $G \in \Sigma \Rightarrow$ es gibt $n \in \{1, 2, …\}$ mit $G \in \Sigma_n$\\
Nach Beh. folgt dass $\# \{ m | f(A_j) = f_m(A_j), 1 \leq j \leq n \} = \infty$\\
also gibt es ein $m \geq n$ mit $f(A_j) = f_m(A_j), 1 \leq j \leq m$\\
$f_m$ ist Modell von $\Sigma_m$ wegen $\Sigma_n^{n \leq m} \subseteq \Sigma_m$ ist $f_m$ auch Modell von $\Sigma_n \Rightarrow f_m(G) = W$
Aber $f_m(G) = f(G) $ da $f(A_j) = f_m(A_j), 1 \leq A_j \leq n \leq m$\\
$\Rightarrow f(G) = W$ $\Box$

\folgerung{3.2}
Sei $\Sigma$ Formelmenge\\
Dann gilt:\\
$\Sigma$ unerfüllbar $\Leftrightarrow$ Es gibt endliche Teilmenge von $\Sigma$, die unerfüllbar ist.

\beweis{} Folgt sofort aus Kompaktheitssatz $\Box$

Alogrithmische Anwendung:\\
Sei $\Sigma$ eine Formelmenge und $\Sigma_n \in \Sigma$ mit
\begin{enumerate}
\item[(1)] $\# \Sigma_n < \infty$

\item[(2)] $\bigcup_{n=1}^{\infty} \Sigma_n = \Sigma$
\end{enumerate}

\begin{enumerate}
\item[(1)] $n=1$

\item[(2)] Ist $\Sigma_n$ unerfüllbar STOP.\\
$\Sigma$ ist unerfüllbar \hspace{1cm} (kann wegen $\# \Sigma_n < \infty$ getestet werden)

Ist $\Sigma_n$ erfüllbar\\
$n=n+1$, gehe zu (2)
\end{enumerate}

Wegen Folgerung 3.2 gilt:\\
Algorithmus terminiert genau dann, wenn $\Sigma$ unerfüllbar ist.