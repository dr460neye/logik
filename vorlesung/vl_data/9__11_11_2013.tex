\datum{11.11.2013}


Kompaktheitssatz\\
$\Sigma$ erfüllbar $\Leftrightarrow$ Jede endl. Teilmenge von $\Sigma$ erfüllbar.\\

\beweis{}
„$\Rightarrow$“\\
„$\Leftarrow$“ $\Sigma_n$ = Formeln aus $\Sigma$ über $A_1, …, A_n$\\
$\Sigma'_n =$ Repräsenten der semantischen Klassen aus $\Sigma_n$\\

Fakt: $\# \Sigma'_n < \infty$\\
Daher $\Sigma'_n$ erfüllbar und es gibt Modell $f_n$

\begin{enumerate}
\item[(1)] $I = \{1, 2, … \}, i=1$

\item[(2)] $\# \{ n \in I | f_n(A_i)=W \} = \infty \rightsquigarrow f(A_i) = W$\\
$I= I \backslash \{ n | f_n(A_i) = F \}$\\
sonst $f(A_i)=F, I:= I \backslash \{ n | f_n(A_i) = W \}$

\item[(3)] $i=i+1$ gehe zu (2)
\end{enumerate}

Wir zeigen: $f$ ist Modell für $G$\\

\behauptung{}
Sei $I$ die Menge der Indizes nach dem $i$-ten Durchlauf von (2)-(3).\\
Dann gilt $\# I = \infty$ und für alle $m \in I$ ist $f_m (A_j) = f(A_j), 1 \leq j \leq i$\\

\beweis{}
Induktion nach $i$\\
I.A.:\\
$i=0$\\
$\# I = \# \{ 1, 2, … \} = \infty$\\
und $f(A_j) = f_m(A_j)$ für $1 \leq j \leq i$ ist leere Bed.\\

I.S.:\\
$i \rightarrow i+1$\\
Sei $I$ die Menge nach dem $(i+1)$-ten Durchlauf und $I'$ die Menge nach dem $i$-ten Durchlauf\\

\underline{1. Fall} $\# \{ n \in I' | f_n(A_{i+1}) = W \} = \infty$\\
$\rightarrow I = I' \backslash \{ n \in I' | f_n(A_{i+1}) = F \}$\\
$= \{ n \in I' | f_n(A_{i+1}) = W \}$\\
$\Rightarrow \# I=\infty$\\

Nach I.V. gilt $f_m(A_j) = f(A_j), 1 \leq j \leq i$\\
und $m \in I'$\\
Da $I \subseteq I'$ folgt\\
$f_m(A_j) = f(A_j), 1 \leq j \leq i$ und $m \in I$\\
Im 1. Fall wird $f(A_{i+1})=W$ und $f_m(A_{i+1})=W$ für $m \in I$\\
$\Rightarrow f(A_j) = f_m(A_j)$ für $1 \leq j \leq i+1$ und $m \in I$\\
\underline{2. Fall} $\# \{ n \in I' | f_n(A_{i+1})=W \} < \infty$\\
$\Rightarrow$ Da nach I.V. $\# I' = \infty \Rightarrow \# \{n \in I' | f_n(A_{i+1}) = F \} = \infty$\\

Beweis nun analog zum 1. Fall\\
Sei $G \in \Sigma \Rightarrow$ es gibt $n \in \{1, 2, …\}$ mit $G \in \Sigma_n$\\
Nach Beh. folgt dass $\# \{ m | f(A_j) = f_m(A_j), 1 \leq j \leq n \} = \infty$\\
also gibt es ein $m \geq n$ mit $f(A_j) = f_m(A_j), 1 \leq j \leq m$\\
$f_m$ ist Modell von $\Sigma_m$ wegen $\Sigma_n^{n \leq m} \subseteq \Sigma_m$ ist $f_m$ auch Modell von $\Sigma_n \Rightarrow f_m(G) = W$
Aber $f_m(G) = f(G) $ da $f(A_j) = f_m(A_j), 1 \leq A_j \leq n \leq m$\\
$\Rightarrow f(G) = W$ $\Box$

\folgerung{3.2}
Sei $\Sigma$ Formelmenge\\
Dann gilt:\\
$\Sigma$ unerfüllbar $\Leftrightarrow$ Es gibt endliche Teilmenge von $\Sigma$, die unerfüllbar ist.

\beweis{} Folgt sofort aus Kompaktheitssatz $\Box$

Alogrithmische Anwendung:\\
Sei $\Sigma$ eine Formelmenge und $\Sigma_n \in \Sigma$ mit:\\
($\Sigma_n \subseteq \Sigma_{n+1}$ für alle $n$)
\begin{enumerate}
\item[(1)] $\# \Sigma_n < \infty$

\item[(2)] $\bigcup_{n=1}^{\infty} \Sigma_n = \Sigma$
\end{enumerate}

\begin{enumerate}
\item[(1)] $n=1$

\item[(2)] Ist $\Sigma_n$ unerfüllbar STOP.\\
$\Sigma$ ist unerfüllbar \hspace{1cm} (kann wegen $\# \Sigma_n < \infty$ getestet werden)

Ist $\Sigma_n$ erfüllbar\\
$n=n+1$, gehe zu (2)
\end{enumerate}

Wegen Folgerung 3.2 gilt:\\
Algorithmus terminiert genau dann, wenn $\Sigma$ unerfüllbar ist.

\newpage

\section{Resolutionen}

Bisher: Test auf Erfüllbarkeit: Semantische Verfahren\\
„suche Modell“\\

Jetzt: Syntaktische Verfahren\\
„Transformation der Formel“\\

Für das \Index{Resolutionsverfahren} nimmt man an, dass die Formel schon in VKNF sind.\\ Die Disjunktionsterme werden dazu noch umgeschrieben:\\

Notation: $\bigvee_{i=1}^l L_i$ Disjunktionsterm $\rightsquigarrow \{ L_1, …, L_l \}$

\definition{}
Eine endliche Menge von Literalen heißt \Index{Klausel}.\\
Eine Menge von Klauseln heißt Klauselmenge

\bemerkung{}
\begin{itemize}
\item Klauseln können das positive und negative Literal zu einer atomaren Formel gleichzeitig enthalten.\\
Disjunktionsterme aber nicht

\item Klauseln sind endliche Mengen\\
Klauselmengen können unendlich sein
\end{itemize}

Notation: Sei $\bigwedge_{i=1}^s \bigvee_{j=1}^{r_i} L_{i,j}$ in VKNF identifizieren diese mit Klauselmenge\\
$\{ \{ L_{1,1}, …, L_{1, r_1} \}, \{ L_{2,1}, …, L_{2, r_2}\}, …, \{ L_{s,1}, …, L_{s, r_s} \} \}$

\definition{}
\begin{itemize}
\item Die leere Klausel wird mit $\Box$ bezeichnet und ist erfüllbar.

\item Eine Klausel $\{ L_1, …, L_l \} + \Box$ ist erfüllbar $\leftrightarrow$ $\bigvee_{i=1}^l L_i$ ist erfüllbar

\item Eine Klauselmenge ist erfüllbar, falls es eine Interpretation $f$ gibt, die alle Klauseln in der Klauselmenge erfüllt.

\item Ein Model einer Klauselmenge ist eine Interpretation $f$, die alle Klauseln in der Klauselmenge erfüllt.
\end{itemize}

\beispiel{}\\
$(A_1 \OR \neg A_4) \AND (A_2 \OR \neg A_3 \OR A_4) \AND ( \neg A_1 \OR A_2 \OR A_3)$ VKNF\\
$\{ \{ A_1, \neg A_3 \}, \{ A_2, \neg A_3, A_4 \}, \{ \neg A_1, A_2, A_3 \} \}$ Klauselmenge\\

$\begin{rcases} f(A_1) = f(A_2) = W \\ f(A_3), f(A_4)\ beliebig \end{rcases}$ ist Modell der Klauselmenge\\

\bemerkung{}
$K$ Klauselmenge und $\Box \in K \Rightarrow K$ ist unerfüllbar

\definition{}
Seien $K_1, K_2$ zwei Klauseln und $L_1 \in K_1, L_2 \in K_2$\\
das positive und negative Literal zu selben atomaren Formel.\\

Dann heißt $R = (K_1 \backslash \{L_1\}) \OR (K_2 \backslash \{L_1\})$ eine \Index{Resolvente} von $K_1$ und $K_2$

\beispiel{}
$K = \{ \underbrace{ \{ A_1, \neg A_4 \} }_{K_1}, \underbrace{ \{ A_2, \neg A_3, A_4 \} }_{K_2}, \underbrace{ \{ \neg A_1, A_2, A_3 \} }_{K_3} \}$

$L_1 = \neg A_4 \in K_1$\\
$L_2 = A_4 \in K_2$\\
$R = (K_1 \backslash \{ L_1 \}) \OR (K_2 \backslash \{L_2\})$ $=\{A_1, A_2, \neg A_3\}$ \\

$L_1 = \neg A_3 \in K_2$\\
$L_2 = A_3 \in K_3$\\
$R = (K_2 \backslash \{L_1\}) \OR (K_3 \backslash \{L_2\}) = \{A_2, A_4, \neg A_1\}$\\

$K = \{ \underbrace{ \{ A_1, \neg A_2 \} }_{K_1}, \underbrace{ \{\neg A_1, A_2 \} }_{K_2} \} \rightsquigarrow^{VKNF} (A_1 \OR \neg A_2) \AND (\neg A_1 \OR A_2)$\\

$\begin{rcases} L_1 A_1 \in K_1 \\ L_2 = \neg A_1 \in K_2 \end{rcases} R= (K_1 \backslash \{L_1\}) \OR (K_2 \backslash \{L_2\}) = \{A_2, \neg A_2\} \rightarrow$ gehört nicht zu Disjunktionsterm

\lemma{4.1} (Resolutionslemma)
Sei $K$ Klauselmenge, $K_1, K_2 \in K$ KLauseln und $R$ Resolvent von $K_1$ und $K_2$\\

Dann gilt
\begin{enumerate}
\item[(1)] $K \models R$

\item[(2)] $K \equiv K \cup \{R\}$
\end{enumerate}

\beispiel{}\\
$K1 = \{A_1, \neg A_4\}$\\
$K_2 = \{ A_2, \neg A_3, \neg A_4\}$\\
$R = \{A_1, A_2, \neg A_3\}$\\
Wollen sehen $\{K_1, K_2\} \models R$\\
$f(K_1) = W \rightarrow f(A_1) = W $ oder $ f(\neg A_4) = W$\\

$f(A_1) = W \Rightarrow f(R)=W$\\
$\begin{rcases} f(\neg A_4) = W \\ f(K_2)=W \end{rcases} f(A_2) = W \\ oder\ f(\neg A_3)$
$\Rightarrow f(R) = W$\\
Also $\{K_1, K_2\} \models R$
