\datum{21.10.2013} 


\satz{1.5} $G \equiv H$\\
$I$ mit $G$ Teilformel\\
$I'$ ersetze ein Vorkommen von $G$ in $I$ durch $H$\\
$\Rightarrow I \equiv I'$

\beispiel{}\\
$\begin{rcases}G = \neg ( \neg A_1 \OR A_3)\\
H = (A_1 \AND \neg A_3) \end{rcases} G \equiv H$\\
\noindent\\
$I = ((( \neg A_{2} \OR A_4) \AND ( A_6 \OR \neg ( \neg A_1 \OR A_3))) \AND \neg ( \neg A_1 \OR A_3))$\\
$I' = ((( \neg A_2 \OR A_4) \AND (A_6 \OR ( A_1 \AND \neg A_3))) \AND \neg ( \neg A_1 \OR A_2))$\\

\beweis{}
Induktion über die Länge  $l = \#$ zeichen von $I$\\
I.A.: $l$ = Länge von $G$\\
$\Rightarrow I = G$ und $I' = H$\\
$\Rightarrow I = G \equiv H = I'$\\
\noindent\\
I.V.: Die Aussage gilt für alle Formeln der Länge $\leq l$\\
I.S.: Sei $I$ Formel der Länge $l+1$, die $G$ als Teilformel enthält\\
\noindent\\
\underline{1. Fall}\\
$I = (I_1 \AND I_2)$ mit Formeln $I_1$ und $I_2$ der Länge $\leq l$

\begin{enumerate}
\item[i)] $I' = (I'_1 \AND I_2)$\\
$I'_1$ entsteht aus $I_1$ durch ersetzen der Teilformel $G$ in $I$, durch $H$\\
Sei Länge von $H_1 \leq l$ folgt nach I.V\\
$I_1 = = I'_1$\\
Also\\
$\begin{rcases} I = (I_1 \AND I_2)$, $I'=(I'_1 \AND I_2) \\
I_1 \equiv I'_1 \\
I_2 = I_2 \end{rcases} I' \equiv I$\\

\item[ii)] $I' = I_1 \AND I'_2$\\
$I'_2$ entsteht aus $I_2$ durch ersetzen der Teilformel $G$ in $I_2$ durch $H$\\
weiter analog zu (i)
\end{enumerate}

\noindent
\underline{2. Fall}\\
$I = (I_1 \OR I_2)$\\
analog zu 1. Fall\\

\noindent
\underline{3. Fall}\\
$I_1 = \neg I_1$ ist Formel $I_1$ der länge $\leq l$\\
$\rightarrow I' = \neg I'_1$ und $I'_1$ entsteht aus $I_1$ durch ersetzen der Teilformel $G$ durch $H$\\

\noindent
Nach I.V. \hspace{1cm} $I_1 \equiv I'_1$\\
$\Rightarrow^{Lemma\ 1.4}$ (iii) $I = \neg I_1 \equiv \neg I'_1 = I'$ $\Box$\\
$\rightarrow$ Der Beweis von Satz 1.5 ist ein erstes Beispiel des Beweisprinzips „\Index{Induktion über Formelaufbau}“\\
(„\Index{Strukturelle Induktion}“)

\behauptung{„Aussage über Formeln“}

\beweis{}
Induktion über Formelaufbau\\
I.A.: Aussage für Kürzest mögliche Formel\\
I.S.: Die Aussage gilt für die 3 Möglichkeiten\\
für $(G \AND H)$, $(G \OR H)$, $\neg G$\\
falls die Aussage für $G$ und $H$ gilt.

\definition{}
Sei $\Sigma$ eine Menge von Formeln\\
$\Sigma$ heißt erfüllbar, falls es eine Interpretation $f$ gibt mit $f(G) = W\ \forall\ G \in \Sigma$\\
so ein $f$ heißt \Index{Modell} $\Sigma$

\noindent
$\Sigma$ heißt unerfüllbar oder nicht erfüllbar, falls es keine Interpretation $f$ gibt mit $f(G) = W$

\noindent
für alle $G \in \Sigma$\\

\beispiel{}
$\Sigma = \{A_1, A_1 \AND A_2, A_1 \AND A_2 \AND A_3, … \}$\\
$= \{A_1 \AND … \AND A_n \ |\ n=1 \}$\\

\noindent
erfüllbar $f(A_i) = W$ für $i=1$ ist einziges Modell\\
$\Sigma = \{A_1 \AND A_2, \neg ( \neg A_1 \OR A_2) \}$ nicht erfüllbar\\

\beweis{}
\begin{enumerate}
\item[i)] $G$ erfüllbar $\Leftrightarrow \{G\}$ erfüllbar
\item[ii)] $\Sigma$ erfüllbar $\Rightarrow G$ erfüllbar für alle $G \in \Sigma$
\item[iii)] $G$ erfüllbar für alle $G \in \Sigma \not \Rightarrow \Sigma$ erfüllbar
\end{enumerate}
$\Sigma = \{A_1, \neg A_1 \}$ unerfüllbar\\
\noindent\\
$A_1, \neg A_1$ erfüllbar wenn $A_1 = W$ bzw. $F$

\lemma{1.6}
Sei $\Sigma = \{G_1, G_n \}$ endlich, dann folgt\\
$\Sigma$ erfüllbar $\leftrightarrow (G_1 \AND … \AND G_n)$ erfüllbar

\beweis{}
$\Sigma$ erfüllbar $\Leftrightarrow$ Es gibt Interpretation $f$ mit $f(G_1) = W$, $i=1, …, n$\\
$\Leftrightarrow$ Es gibt Interpretation $f$ mit $f(G_1 \AND … \AND G_n) = W$\\
$\Leftrightarrow ( G_1 \AND … \AND G_n)$ erfüllbar $\Box$

\definition{}
Eine Formelmenge $\Sigma$ \Index{impliziert semantisch} eine Formel $H$, falls für jedes Modell $f$ von $\Sigma$ gilt $f(H) = W$; d.h. $f$ ist auch Modell von $\{H\}$. Wir schreiben $\Sigma \models H$. Ist $\Sigma = \{G\}$ so schreiben wir auch $G \models H $ für $G$ \Index{impliziert semantisch} $H$.

\beispiel{}
$A_1 \AND A_2 \models A_1 \OR A_2$\\
$f(A_1 \AND A_2) = W \Rightarrow f(A_1) = f(A_2) = W$\\
$\Rightarrow f(A_1 \OR A_2) = W$\\
$A_1 \OR A_2 \not \models A_1 \AND A_2$\\

\noindent
für $f(A_1) = W,\ f(A_2) = F$ gilt $f(A_1 \OR A_2) = W$\\
aber $f(A_1 \AND A_2) = F$\\
falls $G \models H$ und $f(G) = F$ muss nicht $f(H) = f$ gelten.\\
$(A_1 \AND A_2) \models (A_1 \OR A_2)$\\
für $f(A_1) = W,\ f(A_2)=F$ gilt $f(A_1 \AND A_2) = F$\\
aber $f(A_1 \OR A_2 ) = W$

\bemerkung{}
$G, H$ Formeln\\
$G \equiv H \Leftrightarrow ( G \models H)$ und $(H \models G)$\\

\beweis{}
$G \equiv H$, für jede Interpretation $f$ gilt $f(H) = f(G)$\\
$\Leftrightarrow$ für jede Interpretation $f$ gilt\\
$f(G) = W \Leftrightarrow f(H) = W$\\
$\Leftrightarrow G \models H$ und $H \models G$ $\Box$

\beispiel{}
Es gibt Formel $G, H$ mit $G \not \models H, H \not \models G$\\
$G = A_1$, $H=\neg A_1$

\satz{1.7} \textbf{\Index{Craig'scher Interpolationssatz}}\\
Seien $G$ und $H$ Formel mit $G \models H$\\
Dann gilt mindestens eine der folgenden 3 Aussagen:
\begin{itemize}
\item[i)] $H$ ist gültig
\item[ii)] $G$ ist unerfüllbar
\item[iii)] Es gibt eine Formel $I$ mit $G \models I$, $I \models H$\\
und jede atomare Teilformel von $I$ ist atomare Teilformel von $G$ und $H$.
\end{itemize}

\beispiel{}
$G = (A_1 \AND A_2)$, $H = (A_2 \OR A_3)$\\
$G \models I$ \hspace{1cm} $I \models H$ \hspace{1cm} für $I= A_2$\\
$(A_1 \AND A_2) \models A_2$ \hspace{1cm} $A_2 \models (A-2 \OR A_3)$

\beweis{}
Es genügt zu Zeigen\\
$G \models H$ und $H$ nicht gültig mit $G$ erfüllbar $\Rightarrow$ (iii)\\

\noindent
Induktion über die Anzahl der atomaren Teilformeln von $G$, die nicht in $H$ vorkommen.\\
I.A: $n=0$ \hspace{1cm} Können $I = G$ wählen\\
$G \models I = G$ \hspace{1cm} $G = I \models H$\\

