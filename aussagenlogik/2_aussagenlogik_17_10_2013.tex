
\datum{17.10.2013}

\definition{}
Sei $F_n$ die Menge aller Formeln über $\{A_1, …, A_n\}$ eine Teilmenge $K \subseteq F_n$ heißt semantische Klasse, falls $G \equiv H$ für $G, H \in K$ und $G \not\equiv H$ für $G\in K$ \hspace{.5cm} $H \in F_n \backslash K$\\

\beispiel{}
$F_1 = \{A_1, \neg A_1, A_1 \OR A_1, A_1 \AND A_1, … \}$\\

\begin{minipage}{0.15\textwidth}
\begin{tabular}{c|c}
$A_1$ & $A_1$\\
\hline
F & F\\
W & W\\
\end{tabular}
\end{minipage}
\hfill
\begin{minipage}{0.15\textwidth}
\begin{tabular}{c|c}
$A_1$ & $\neg A_1$ \\
\hline
F & W\\
W & F\\
\end{tabular}
\end{minipage}
\hfill
\begin{minipage}{0.15\textwidth}
\begin{tabular}{c|c}
$A_1$ & $A_1 \OR A_1$\\
\hline
F & F\\
W & W\\
\end{tabular}
\end{minipage}
\hfill
\begin{minipage}{0.15\textwidth}
\begin{tabular}{c|c}
$A_1$ & $A_1 \OR \neg A_1$\\
\hline
F & W\\
W & F\\
\end{tabular}
\end{minipage}
\hfill
\begin{minipage}{0.15\textwidth}
\begin{tabular}{c|c}
$A_1$ & $A$\\
\hline
F & F\\
W & F\\
\end{tabular}
\end{minipage}\\
\noindent\\
$A_1 \equiv A_1 \OR A_1$\\
\noindent\\
Das sind alle Wahrheitstabellen, die bei einer atomaren Formel möglich sind.\\
\noindent\\
Die semantischen Klassen sind:\\
$K_1 = \{G \in F_1 | G \equiv A_1\}$\\
$K_2 = \{G \in F_1 | G \equiv \neg A_1\}$\\
$K_3 = \{G \in F_1 | G \equiv A_1 \OR \neg A_1\}$\\
$K_4 = \{G \in F_1 | G \equiv A_1 \AND \neg A_1\}$\\
    
\bemerkung{}
„$\equiv$“ ist Äquivalenzrelation auf $F_n$ und die semantischen Klassen sind die Äquivalenzklasse bzgl „$\equiv$“\\
    
\bemerkung{}
\begin{itemize}
\item Die Elemente einer semantischen Klasse sind alle Formeln mit der gleichen Wahrheitstabellen
\item Jede semantische Klasse enthält unendlich viele Formeln\\
    $(G \equiv \neg \neg G \equiv \neg \neg \neg \neg G \equiv … )$
\end{itemize}

\definition{}
\begin{itemize}
\item Eine Formel $G$ heißt gültig oder Tautologie, falls $f(G) = W$ für jede Interpretation $f$
\item Eine Formel $G$ heißt erfüllbar, falls es eine Interpretation $f$ gibt mit $f(G) = W$
\item Eine Formel $G$ heißt nicht erfüllbar, unerfüllbar oder Antilogie, falls $f(G) = F$ für alle Interpretationen $f$
\end{itemize}

\beispiel{}\\
$\begin{rcases} A_1 \OR \neg A_1\ g\ddot{u}ltig \\ A_1 \\ \neg A_1 \end{rcases} erf\ddot{u}llbar$\\
$A_1 \AND \neg A_1$ unerfüllbar\\

\noindent\\
Ein zentrales Ziel der Vorlesung wird es sein, Algorithmen zu entwickeln, die zur Gegebenen Formel $G$ entscheiden, da diese erfüllbar ist.\\
    
\noindent\\
\bemerkung{}
\begin{enumerate}[i)]
\item $G$ gültig $\Rightarrow G$ ist er erfüllbar
\item $G$ ist unerfüllbar $\Leftrightarrow \neg G$ gültig
\item $G, H$ gültig $\Rightarrow G \equiv H$
\item $G, H$ nicht erfüllbar $\Rightarrow G \equiv H$
\end{enumerate}

\noindent
Wir untersuchen nun die Ausdruckskraft der Aussagenlogik:\\
Gegeben eine Wahrheitstabelle\\
z. B.\\
\begin{tabular}{c|c|c|c}
$A_1$ & $A_2$ & $A_3$ & ?\\
\hline
F & F & F & W\\
F & F & W & F\\
F & W & F & F\\
W & F & F & W\\
F & W & W & F\\
W & F & W & F\\
W & W & F & W\\
W & W & W & F
\end{tabular}\\
Gibt es Formel $G$ mit dieser Wahrheitstabelle?\\

\lemma{1.2}
Über eine Menge von $n$ atomaren Formeln gibt es $2^{2^{n}}$ semantische Klassen. Insbesondere gibt es zu jeder Wahrheitstabelle eine Formel mit dieser Wahrheitstabelle.\\
    
\beweis{}
$\rightarrow$ Zeige: Es gibt $2^{2^n}$ Wahrheitstabellen über $n$ atomare Formeln. Die $2^n$ Interpretationen (Lemma 1.1) legen die Wahrheitstabelle bis auf die letzte Spalte fest.\\
Für jede Zeile gibt es zwei Möglichkeiten für den Eintrag in die letzte Spalte\\
\noindent\\
$\underbrace{2 \cdot 2 \cdot … \cdot 2}_{2^n} = 2^{2^n}$ Möglichkeiten\\
    
\noindent\\
$\rightarrow$ Zeige: Zu jeder Wahrheitstabelle gibt es Formel mit dieser Wahrheitstabelle\\
    
\noindent\\
Konstruktion\\

\begin{tabular}{c|c|c|c}
$A_1$ &  & $A_n$ &  \\
\hline
$f_1 F$ & … & $F$ & $W_1$\\
\vdots   &  & \vdots & \vdots \\
$f_n W$ & … & $W$ & $W_{2^n}$ \\
\end{tabular}
\hspace{1cm}$W_1, …, W_{2^n} \in \{W, F\}$\\
    
\noindent\\
Seien $i_1, …, i_l$ die Indizes mit $W_{i_j} = W$ für $1 \leq j \leq l$\\
$G_k = \bigwedge_{m=1}^n H_{k,m}$, $H_{k,m} = \begin{cases} A_m\ falls\ f_{i_k}\ (A_m)=W \\ \neg A_m\ falls\ f_{i_k}\ (A_m)=F \end{cases}$\\
    
\noindent\\
Setz $G = \bigvee_{k=1}^l G_k$\\
    
\noindent\\
Sei $f$ Interpretation\\
\begin{tabbing}
$G(G_k) = W$ \= $\Leftrightarrow f(H_{2,m}) = W\ für\ m=1, …, m$\\
\> $\Leftrightarrow f(f_m) = W\ für\ W\ mit\ f_{i,k}(A_m) = W$\\
\> = $f(\neg A_m) = W\ für\ W\ mit\ f_{i,k}(A_m) = F$\\
\> $\Leftrightarrow f = f_{i,k}$\\
\end{tabbing}

\noindent\\
\begin{tabbing}
$f(G) = W$ \= $\Leftrightarrow \exists k : f(G_k) = W$\\
\> $\Leftrightarrow \exists k: f= f_{i,k}$\\
\end{tabbing}
$\Rightarrow G$ hat die gegebene Wahrheitstabelle\\

\beispiel{}\\
\begin{tabular}{cc|c|c|c}
 &  & $A_1$ & $A_2$ &   \\
\hline
$\rightarrow$ & $f_1$ & F & F & \textcircled{W}\\
$\rightarrow$ & $f_2$ & W & F & \textcircled{W}\\
 & $f_3$ & F & W & F\\
 & $f_4$ & W & W & F\\
\end{tabular}\\
(suche Zeilen, die hinten $W$ sind)\\

\noindent
$i_1 = 1, i_2 = 2, l=2$\\
$G_1 = (\neg A_1 \AND \neg A_2)$\\
$G_2 = (A_1 \AND \neg A_2)$\\
$G = G_1 \OR G_2 = (( \neg A_1 \AND \neg A_2) \OR (A_1 \AND \neg A))$\\

\vspace{1cm}
\noindent
Regeln zur semantischen Äquivalenz.

\satz{1.3} Seien $G, H, I$ Formeln der Aussagenlogik\\
Dann gilt
\begin{tabbing}
Idempotenz \hspace{1cm} \= $(G \AND G) \equiv G$\\
\> $(G \OR H) \equiv (H \OR G)$\\
%\end{tabbing}
\\
%\begin{tabbing}
Assoziativität \> $((G \AND H) \AND I ) \equiv (G \AND (H \AND I))$\\
\> $((G \OR H) \OR I) \equiv (G \OR (H \OR I ))$\\
%\end{tabbing}
\\
%\begin{tabbing}
Distributivität \> $(G \AND ( H \OR I )) \equiv ((G \AND H ) \OR (G \AND I ))$\\
\> $(G \OR ( H \AND I )) \equiv (( G \OR H ) \AND ( G \OR I ))$\\
%\end{tabbing}
\\
%\begin{tabbing}
Absorption \> $(G \AND ( G \OR H )) \equiv G$\\
\> $(G \OR ( G \AND H )) \equiv G$\\
%\end{tabbing}
\\
Doppelnegation \> $\neg \neg G \equiv G$\\
\\
%\begin{tabbing}
deMorgan Regel \> $\neg ( G \AND H ) \equiv \neg G \OR \neg H$\\
\> $\neg ( G \OR H) \equiv \neg G \AND \neg H$
\end{tabbing}

\beweis{}
Beispielhaft Absorption

\begin{tabular}{c|c|c|c}
$G$ & $H$ & $G \AND ( G \OR H)$ & $G$ \\
\hline
F & F & F & F \\
F & W & F & F \\
W & F & W & W \\
W & W & W & W 
\end{tabular}
$\Rightarrow (G \AND ( G \OR H )) \equiv G$

\lemma{1.4}
$G, H, G', H'$ Formeln der Aussagenlogik\\
und $G \equiv G'$ und $H \equiv H'$\\
\noindent\\
$\Rightarrow$\\
$(G \AND H) \equiv (G' \AND H')$\\
$(G \OR H) \equiv (G' \OR H')$\\
$\neg G \equiv \neg G'$

\beweis{}\\
\begin{tabular}{c|c|c|c|c|c}
$G$ & $H$ & $G'$ & $H'$ & $G \AND H$ & $G' \AND H'$ \\
\hline
F & F & F & F & F & F \\
F & W & F & W & F & F \\
W & F & W & F & F & F \\
W & W & W & W & W & W \\
\end{tabular}
$\Rightarrow (G \AND H ) \equiv ( G' \AND H')$\\
\noindent\\
andere analog

\satz{1.5}
Seien $G, H, I$ Formeln der Aussagenlogik und $G \equiv H$ und $G$ ist Teilformel von $I$.\\
Sei $I'$ eine Formel, die aus $I$ entsteht indem man ein Vorkommen von $G$ durch $H$ ersetzt\\
Dann $I \equiv I'$