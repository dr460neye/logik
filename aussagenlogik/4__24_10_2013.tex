\datum{24.10.2013}


\satz{1.7} \textbf(Graig'scher Interpolationssatz)\\
Seien $G,H$ Formeln mit $G \models H$\\
Dann gilt mindestens eine der folgeden Aussagen:\\
\begin{enumerate}
\item[i)] $H$ gültig
\item[ii)] $G$ unerfüllbar
\item[iii)] Es gibt eie Formel $I$ mit den Eigenschaften $G \models I$, $I \models H$\\
und jede atomare Teilformel von $I$ ist Teilformel von $G$ und $H$.
\end{enumerate}

\beweis{}
Es genügt zu zeigen:\\
%$\begin{rcases}$
\begin{itemize}
\item $H$ nicht gültig
\item $G$ erfüllbar
\item $G \models H$
\end{itemize}
%$\end{rcases}$
$\Rightarrow$ iii)\\

\noindent
Induktion über die Anzahl der atomaren Teilformeln von $G$, die nicht in $H$ enthalten sind.\\

\noindent
I.A.: Sei $n=0$. Dann ist jede atomare Teilformel von $G$ in $H$ enthalten.\\
Wähle $I=G$, dann enthält $I$ nur atomare Teilformeln, die in $G$ und $H$ enthalten sind und $G \models I$ und $I \models H$.\\

\noindent
I.V.:
\begin{itemize}
\item $G \models H$ 
\item $H$ nicht gültig
\item $G$ erfüllbar
\item Die Anzahl der atomaren Teilformeln von $G$, die nicht in $H$ enthalten sind, ist $n$.
\end{itemize}
$\Rightarrow$ iii)\\

\noindent
I.S.: $(n \rightarrow n+1)$\\
\begin{itemize}
\item $G \models H$
\item $H$ nicht gültig
\item $G$ erfüllbar
\item Die Anzahl der atomaren Formeln von $G$, die nicht in $H$ enthalten sind, ist $n+1$
\end{itemize}
\noindent
Sei $A_i$ eine atomare Formel, die in $G$ aber nicht in $H$ enthalten ist.\\
Sei $G_W$/$G_F$ die Formel, die aus $G$ entsteht, wenn jedes Vorkommen von $A_i$ ersetzt wird durch eine gültige/ungültige Formel, die nur atomare Formeln von $G$ und $H$ verwendet.\\

\noindent
Setze $G' := (G_W \OR G_F)$
\begin{enumerate}
\item[1)] Es gibt eine atomare Formel, die in $G$ und in $H$ enthalten ist.\\
(d.h. $G'$ ist wohldefiniert)\\
\underline{Beweis:} Angenommen $G$ und $H$ besitzen keine gemeinsame atomare Formel.\\
Da $G$ erfüllbar ist, gibt es eine Interpretation $f_G$ der atomaren Teilformeln von $G$ mit $f_G (G) = W$\\
Da $H$ nicht gültig ist, gibt es eine Interpretation $f_H$ der atomaren Teilformeln von $H$ mit $f_H (H) = F$\\
Da die Menge der atomaren Teilformeln von $G$ disjunkt zur Menge der atomaren Teilformeln von $H$ ist, gibt es eine gemeinsame Erweiterung $f$ von $f_G$ und $f_H$\\
Damit ist $f$ eine Interpretation der atomaren Teilformeln von $G$ und $H$ mit:
\begin{itemize}
\item $f(G) = f_G(G) = W$
\item $f(H) = f_H(H) = F$\\ \widerspruch zu $G \models H$
\end{itemize}
$\Rightarrow$ Ann. falsch $\Rightarrow$ Beh.

\item[2)] \underline{Beh:} $G' \models H$ \\
\underline{Beweis:} Es genügt zu zeigen:
\begin{enumerate}
\item[a)] $G_W \models H$
\item[b)] $G_F \models H$
\end{enumerate}

zu a)\\
Sei $f$ eine Interpretation der atomaren Teilformeln $G_W$ und $H$ mit $f(G_W) = W$.\\
Erweitere $f$ zu einer Interpretation $f'$ der atomaren Formeln von $G$ und $H$ durch\\
$f'(A_j) = \begin{cases} f(A_j),\ falls\ A_j \neq A_i \\ W,\ falls\ A_j = A_i \end{cases}$\\
$\Rightarrow f'(G) = f(G_W) = W$\\
$\Rightarrow^{(G \models H)} f(H) = f'(H) = W$\\
$\Rightarrow G_W \models H$\\

zu b)\\
Sei $f$ eine Interpretation von $G_F$ und $H$ mit $f(G_F) = W$.\\
Erweitere $f$ zu einer Interpretation $f'$ von $G$ und $H$ durch\\
$f'(A_j) = \begin{cases} f(A_j),\ falls\ A_j \neq A_i \\ F,\ falls\ A_j = A_i \end{cases}$\\
$\Rightarrow^{(G \models H)} f(H) = f'(H) = W$\\
$\Rightarrow G_F \models H$

\item[3)] \underline{Beh:} Die Anzahl der atomaren Formeln in $G'$, die nicht in $H$ enthalten sind ist $n$.\\
\underline{Beweis:} klar

\item[4)] \underline{Beh:} $G'$ ist erfüllbar.\\
\underline{Beweis:}\\
Sei $f$ eine Interpretation der atomaren Formeln in $G$ mit $f(G) = W$.\\
($G$ erfüllbar)\\
Entweder $f(A_i) = W$, d.h. $f(G_W) = W$\\
oder $f(A_i) = F$, d.h. $f(G_F) = W$\\
Somit ist $f(G') = f(G_W \OR G_F) = W$

\item[5)] \underline{Beh:} Es gibt ein $I$, wie in der Beh. des Satzes.\\
\underline{Beweis:} Nach I.V. gibt es eine Formel $I$, welche nur die atomare Formeln von $G'$ und $H$ verwendet mit $G' \models I$ und $I \models H$.

\begin{itemize}
\item $I$ verwendet nur die atomaren Formeln von $G$ und $H$
\item $I \models H$ 
\item zz: $G \models I$\\
Sei $f$ eine Interpretation der atomaren Aussagen von $G$ und $H$ mit $f(G) = W$\\
Dann gilt entweder\\
$f(A_i) = W$ und $f(G_W) = W$\\
oder\\
$f(A_i) = F$ und $f(G_F) = W$\\
Somit folgt $f(G_W \OR G_F) = W$\\
$\Rightarrow^{(G' \models H)} f(I) = W$\\
$\Rightarrow G \models H$ $\Box$
\end{itemize}

\end{enumerate}


\section{Normalformen}

\beispiel{}
$\rightarrow (A_1 \OR \neg A_2 ) \equiv ( \neg A_1 \AND \neg \neg A_2) \equiv ( \neg A_1 \AND A_2)$\\

\definition{}
Ein \underline{Literal} ist eine atomare Formel oder deren Negation. \\

\noindent
Sprechweisen:\\
$A_i \rightarrow$ atomare Formel $\rightarrow$ positives Literal\\
$\neg A_i \rightarrow$ Negation einer atomaren Formel $\rightarrow$ negatives Literal\\

\definition{} Sei $\{A_1, …, A_n\}$ eine Menge von atomaren Formeln\\

\noindent
Ein \underline{Minterm} über $\{A_1, …, A_n\}$ ist eine Konkunktion \\
$L_1 \AND … \AND L_n$ von Literalen $L_i = \begin{cases} A_i \\ \neg A_i \end{cases} , 1 \leq i \leq n$\\
Ein \underline{Maxterm} über $\{A_1, …, A_n\}$ ist eine Disjunktion\\
$L_1 \OR … \OR L_n$ von Literalen $L_i = \begin{cases} A_i \\ \neg A_i \end{cases} , 1 \leq i \leq n$\\

\beispiel{}\\
Sei $\{A_1\}$ Menge von atomaren Formeln\\
Minterme: $A_1$, $\neg A_1$\\
Maxterme: $A_1$, $\neg A_1$\\

\noindent
Sei $\{ A_1, A_2 \}$ Menge von atomaren Formeln\\
Minterme: $(A_1 \AND A_2)$, $(A_1 \AND \neg A_2)$, $(\neg A_1 \AND A_2)$, $\neg ( A_1 \AND \neg A_2)$\\
Maxterme: analog $( \OR )$, $( \OR )$, $( \OR )$, $( \OR )$\\

\lemma{2.1}
Über $\{A_1, …, A_2\}$ gibt es \\
genau $2^n$ Minterme und\\
genau $2^n$ Maxterme.

\beweis{}
Für jedes Literal $L_i$ gibt es (siehe Def.) 2 Möglichkeiten.\\
Also insgesamt $\underbrace{2 \cdot … \cdot 2}_{n-mal} = 2^n$ Möglichkeiten\\

\bemerkung{}
Da $2^n < 2^{2^n}$ für $n \geq 1$ gibt es nicht in jeder semantischen Klasse einen Minterm bzw. Maxterm

