\documentclass[a4paper]{scrartcl}
\usepackage[utf8]{inputenc}
\usepackage[T1]{fontenc}
\usepackage{amsmath}
\usepackage{hyperref}

\setlength\parindent{0pt}

\begin{document}

\title{Erläuterung Markierungsalgorithmus}
\subtitle{(Der Markierungsalgorithmus funktioniert nur bei Horn-Formeln)}
\maketitle

\begin{enumerate}
\item Bei einer Klausel $\{A\}$ schreibt man\\
$W \rightarrow A$

\item Bei einer Klausel $\{A_1, \neg A_2, \neg A_3 … \}$ schreibt man\\
$(A_2 \wedge A_3 \wedge … ) \rightarrow A_1$

\item Bei $\{\neg A\}$ schreibt man\\
$A \rightarrow F$
\end{enumerate}

Jetzt kann es losgehen mit markieren:

\begin{enumerate}
\item zuerst werden alle atomaren Formeln markiert, die als $W \rightarrow A$ vorkommen.\\
Wenn man also $W \rightarrow A$ irgendwo stehen hat, dann muss man alle Vorkommen von $A$ in allen umgeschriebenen Formeln markieren

\item nun schaut man, ob eine Formel da ist mit $A \rightarrow F$, wo $A$ markiert ist\\
bzw. $A_1 \wedge A_2 \wedge … ) \rightarrow F$ und alle Formeln sind markiert\\
wenn so eine da ist, dann bricht der Algorithmus ab und sagt: \textbf{Klauselmenge unerfüllbar}.

Falls keine solche Formel vorhanden, dann weiter mit:

\item suche Formel mit $(A_1 \wedge A_2 \wedge …) \rightarrow A$, wo alle atomaren Formeln auf der linken Seite markiert sind und $A$ noch nicht.

\begin{itemize}
\item falls es so etwas gibt, markiert man alle Vorkommen von $A$ in allen umgeschriebenen Formeln und geht wieder zum vorherigen Schritt - also schauen, ob es eine Formel mit $(A_1 \wedge …) \rightarrow F$ gibt, usw.

\item falls aber nichts neues mehr markiert werden konnte, dann bricht der Alogrithmus ab und sagt: \textbf{Formel erfüllbar}\\
und als Geschenk gibt es nach dem Algorithmus sogar noch ein \textbf{minimales Modell} für die Formel, falls sie erfüllbar ist, nämlich die Interpretation, die jede markierte Formel als wahr und jede nicht markierte als falsch interpretiert
\end{itemize}

\end{enumerate}

\href{http://downloads.mennicke.info/theory/ErfuellbarkeitstestHornformel.html}{Ein Java-Applet zum Markierungsalgorithmus}

\end{document}