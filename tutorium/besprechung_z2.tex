\documentclass[a4paper]{scrartcl}
\usepackage[utf8]{inputenc}
\usepackage[T1]{fontenc}
\usepackage[german]{babel}
\usepackage{amsmath}
\usepackage{amssymb}

\begin{document}

\title{Besprechung Zettel 2}
\subtitle{Tutorium vom 31.10.2013}
\author{Mitschrift aus dem Tutorium (Jens Kosiol), vermutlich mit Fehlern}
\maketitle

\section*{Aufgabe 1}

$G$, $H$ Formeln\\
zz: $G \equiv H \Leftrightarrow (G \leftrightarrow H)$ gültig.\\
$(G \leftrightarrow H)$ gültig. $\Leftrightarrow$ $(G \rightarrow H)$ und $(H \rightarrow G)$ gültig.\\

\begin{tabular}{c|c|c|c}
$G$ & $H$ & $G \rightarrow H$ & $H \rightarrow G$ \\
\hline
F & F & W & W \\
F & W & W & F \\
W & F & F & W \\
W & W & W & W \\
\end{tabular}\\

\noindent\\
Die Hinrichtung „$\Rightarrow$“:\\
$G \equiv H$\\
$\Rightarrow (G \rightarrow H), (H \rightarrow G)$ gültig\\

\noindent\\
Die Rückrichtung „$\Leftarrow$“:\\
$(G \rightarrow H)$\\
$\Rightarrow f(G) = f(H)$\\
$\Leftrightarrow G \equiv H$

\newpage

\section*{Aufgabe 2}

Beh.: $G$ enhält keine Negation $\Rightarrow$ $G$ nicht gültig, aber erfüllbar\\
Bew: Seien $f_W$ und $f_F$ zu $G$ passende Interpretationen mit $f_W(A_i) = W, f_F(A_i) = F$\\
$\forall i \in \{1, …, n\}$\\

\noindent
Wir zeigen $F_W (G) = W, f_F (G) = F$ $\forall G$\\
I.A: Sei $A_i$ atomar, dann gilt:\\
$f_W (A_i) = W, f_F (A_i) = F\ \forall i \in \{1, …, n\}$\\
I.S:
\begin{enumerate}
\item[1. Fall] $G = (G_1 \vee G_2 )$\\
Nach I.V. gilt\\
$f_W(G_1) = W = f(G_2)$\\
$f_F(G_1) = F = f_F(G_2)$\\
Damit gilt\\
$f_W(G) = f_W(G_1 \vee G_2) = W$\\
$f_F(G) = f_F ( G_1 \vee G_2 ) = F$

\item[2. Fall] $G = (G_1 \wedge G_2)$\\
Nach I.V. gilt\\
$f_W (G_1) = f_W (G_1) = W$\\
$f_F (G_1) = f_F (G_2) = F$\\
Damit gilt:\\
$f_W(G) = f_W ((G_1 \wedge G_2)) = W$\\
$f_F(G) = f_F (( G_1 \wedge G_2)) = F$

\item[3. Fall] entfällt
\end{enumerate}
$\Rightarrow$ $G$ ist nicht gültig, aber erfüllbar.

\newpage
\section*{Aufgabe 3}

\subsection*{a)}

Beh.: „$\equiv$“ ist Äquivalenzrelation
\begin{itemize}
\item Reflexivität\\
$G$ Formel. Dann gilt für jede zu $G$ passende Interpretation $f$: $f(G) = f(G)$\\
$\Rightarrow G \equiv G$

\item Symmetrie:\\
$G, H$ Formeln mit $G \equiv H$. Dann gilt für jede zu $G$ und $H$ passende Interpretation $f$: $f(G) = f(H)$ $\Leftrightarrow$ $f(H) = f(G)$\\
$\Rightarrow H \equiv G$

\item Transitivität:\\
$G, H, I$ Formel mit $G \equiv H, H \equiv I$\\
für jede zu $G$ und $H$ passende Interpretation\\
$f$ gilt $f(G) = f(H)$ und jede zu $H$ und $I$ passende Interpretation $g$ mit $g(H) = g(I)$\\
Sei $h$ zu $G$ und $I$ passende Interpretation.\\
Sei $h'$ zu $H$ passend mit $h' = h$ auf $((A \cap T(G)) \cup (A \cap T(I)))$ und beliebig sonst.\\
Dann gilt: $\underbrace{ h' (G) }_{=h(G)} = h' ( H) = \underbrace{ h'(I) }_{= h(I)}$\\
$\Rightarrow G \equiv I$
\end{itemize}

Mit gleichen Wahrheitstabellen kommt man nicht immer weiter.\\
Oder Tabellen, wo ALLE verwendeten atomaren Formeln vorkommen.


\subsection*{b)}



\subsection*{c)}

\paragraph*{ii)}
Beh. $(( G \wedge H) \wedge I) \equiv (G \wedge ( H \wedge I))$\\
$f ((( G \wedge H) \wedge I)) = \begin{cases} W\ falls\ f(G \wedge H), f(I)\ W \\ F\ sonst \end{cases}$\\
$\Leftrightarrow \begin{cases} W\ falls\ f(G), f(H), f(I)\ W \\ F\ sonst \end{cases}$\\
$\Leftrightarrow \begin{cases} F\ sonst \\ W\ falls\ f(G), f(H \wedge I)\ W \end{cases}$\\
$= f((G \wedge (H \wedge I)))$

\newpage
\section*{Aufgabe 4}

Ein Alphabet nur mit $\{\neg, \#, (, ), A_1 … \}$\\
Formel $G$, die nur ein Atom enthält\\
\begin{tabular}{cc|c}
 & $f(A)$ & $f(G_1)$\\
 \hline
$f_2$ & F & W \\
$f_1$ & W & W \\
\end{tabular}

\begin{tabular}{c|c}
$f(A_)$ & $f(G_1)$\\
\hline
F & F \\
W & F \\
\end{tabular}

$f_1(G_1) \neq f_2(G_1)$\\

I.A. $G = A$ atomar\\
$f_1(G_1) = f_1(A) \neq f_2(A) = f_2(G)$\\
IS:
\begin{enumerate}
\item[1 Fall] $G = \neg G_1$\\
$f_1( \neg G_1) \neq f_2( \neg G_1)$

\item[2 Fall] $G = \# (G_1, G_2, G_3)$\\
$f(G_1) = f_1(G_2) = f_1 (G_3) = W$\\
$f_1 (G) = W$\\
$f_2(G) = F$\\
da nach I.V. $f_1(G_i) \neq f_2 (G_i)$

\begin{enumerate}
\item[2.2] Genau $2$ $G_i$ sind wahr unter $f_1$\\
$\Rightarrow f_1(G) = W$\\
$\Rightarrow$ nach IV ist unter $f_2$ nur ein $G_i$ wahr\\
$f_2(G) = F$

\item[2.3] Genau ein $G$ ist $W$ unter $f_1$\\
$\Rightarrow f_1 (G) = F$\\
nach I.V. sind unter $f_2$ genau $2$ $G_i$ $W$\\
$\Rightarrow f_2 (G) = W$\\

\item[2.4] Unter $f_1$ sind alle $G_i$ $f$\\
$f_1(G) = F$ nach IV sind alle $G_i$ $W$ unter $f_2 \Rightarrow f_2 ( G) = W$
\end{enumerate}
\end{enumerate}

\end{document}