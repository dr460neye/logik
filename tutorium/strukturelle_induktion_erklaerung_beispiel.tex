\documentclass[a4paper]{scrartcl}
\usepackage[utf8]{inputenc}
\usepackage[T1]{fontenc}
\usepackage[german]{babel}
\usepackage{amsmath}
\usepackage{amssymb}

\setlength{\parindent}{0pt} 

\begin{document}

\title{Erklärung Strukturelle Induktion}
\subtitle{Mitschrift aus dem Tutorium von Jens Kosiol}
\maketitle

\textbf{Beispiel:} Induktion in $\mathbb{N}$:\\
Wir haben einen „Anfang“ 1\\
Wir haben \underline{eine} Regel, aus Zahlen neue Zahlen zu berechnen: $n+1$\\


\textbf{Strukturelle Induktion:}\\
Wir haben „Anfänge“ $A_1, A_2, …, $, die atomaren Formeln.\\
Wir haben \underline{drei} Regeln, um aus Formeln neue Formeln zu bilden.\\

Idee: Zeige für eine Aussage $P$ dass $P$ für alle atomaren Formeln gilt.\\

Zeige, dass $P$ beim bilden neuer Formeln durch die drei Regeln erhalten bleibt:\\

\underline{Also:} Wenn $P$ für $b_1, b_2$ gilt, dann auch für\\
$\neg G_1$\\
$(G_1 \wedge G_2)$\\
$(G_1 \vee G_2)$\\

\newpage

\textbf{Beispiel:}\\
Blatt 1, 4a (Hier kommen alle drei Fälle vor)\\
$P$: „In jeder semantischen Klasse liegt Formel $G$, die nur $\wedge$ und $\neg$ benutzt.“\\
\underline{I.A.:} $P$ gilt für alle atomaren Formeln.\\
Sei $A_i$ beliebige atomare Formel.\\
Dann $A_i \equiv A_i$ und $A_i$ verwendet $\wedge$ und $\neg$\\
\underline{I.V.:} Seien $G_1$ und $G_2$ Formeln, für die es semantisch äquivalente Formeln $G'_1 \equiv G_1$ und $G'_2 \equiv G_2$ gibt, die nur $\wedge$ und $\neg$ benutzen.\\
\underline{I.S.:} Wir müssen zeigen: Die Eigenschaft $P$ bleibt beim Bilden neuer Formeln erhalten.\\

\begin{enumerate}
\item[1. Fall:] $G = \neg G_1$\\
($G'_x$ vorausgesetzt) Nach I.V. existiert $G'_1 \equiv G_1 \Rightarrow \neg G'_1 \equiv G$\\
und $\neg G'_1$ verwendet nur $\wedge$ und $\neg$. $\checkmark$

\item[2. Fall:] $G = (G_1 \vee G_2)$\\
nach I.V. existiert $G_1 \equiv G'_1$ und $G_2 \equiv G'_2$\\
für die gilt: $G'_1$ und $G'_2$ verwenden nur $\wedge$ und $\neg$\\
$\Rightarrow G = (G'_1 \vee G'_2) \equiv \neg \neg (G'_1 \vee G'_2)$\\
$\equiv \neg ( \neg G'_1 \wedge \neg G'_2)$\\
und $\neg ( \neg G'_1 \wedge G'_2)$ verwendet nur $\wedge$ und $\neg$

\item[3. Fall:] $G = (G_1 \wedge G_2)$\\
Nach I.V. existiert $G'_1 \equiv G_1$ und $G'_2 \equiv G_2$,\\
die nur $\wedge$ und $\neg$ verwenden
\end{enumerate}

Damit $G \equiv (G'_1 \wedge G'_2)$ und $G'_1 \wedge G'_2)$ verwendet nur $\wedge$ und $\neg$

\end{document}